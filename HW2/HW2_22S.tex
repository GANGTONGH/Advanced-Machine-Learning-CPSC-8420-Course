\documentclass[11pt]{article}
\usepackage{amsmath,amsbsy,amssymb,verbatim,fullpage,ifthen,graphicx,bm,amsfonts,amsthm,url}
\usepackage{graphicx}
\usepackage{xcolor}
\newcommand{\mfile}[1]  {{\small \verbatiminput{./#1}}} % Jeff Fessler, input matlab file
\newcommand{\tmop}[1]{\ensuremath{\operatorname{#1}}}
\newcommand{\R}{\mathbb{R}}
\newcommand{\C}{\mathbb{C}}
\newcommand{\Z}{\mathbb{Z}}
\newcommand{\A}{\mathcal{A}}
\newcommand{\minimize}{\operatorname*{minimize\ }}
\newcommand{\maximize}{\operatorname*{maximize}}
\newcommand{\opdet}[1]{\operatorname{\textbf{det}}\left(#1\right)}
\newcommand{\optr}[1]{\operatorname{\textbf{tr}}\left(#1\right)}
\newcommand{\answer}[2][blue]{\ifdefined\AnswerDefine{\color{#1}\it#2}\fi}
\newcommand{\mtx}[1]{\mathbf{#1}}
\newcommand{\vct}[1]{\mathbf{#1}}
\def \lg       {\langle}
\def \rg       {\rangle}
\def \mA {\mtx{A}}
\def \mI {\mtx{I}}
\def \mJ {\mtx{J}}
\def \mU {\mtx{U}}
\def \mS {\mtx{S}}
\def \mV {\mtx{V}}
\def \mW {\mtx{W}}
\def \mLambda {\mtx{\Lambda}}
\def \mSigma {\mtx{\Sigma}}
\def \mX {\mtx{X}}
\def \mY {\mtx{Y}}
\def \mZ {\mtx{Z}}
\def \zero     {\mathbf{0}}
\def \vzero    {\vct{0}}
\def \vone    {\vct{1}}
\def \vu {\vct{u}}
\def \vv {\vct{v}}
\def \vx {\vct{x}}
\def \vy {\vct{y}}
\def \vz {\vct{z}}
\def \vphi {\vct{\phi}}
\def \vmu {\vct{\mu}}
\def \R {\mathbb{R}}


\usepackage{xspace}
\makeatletter
\DeclareRobustCommand\onedot{\futurelet\@let@token\@onedot}
\def\@onedot{\ifx\@let@token.\else.\null\fi\xspace}

\def\eg{\emph{e.g}\onedot} \def\Eg{\emph{E.g}\onedot}
\def\ie{\emph{i.e}\onedot} \def\Ie{\emph{I.e}\onedot}
\def\cf{\emph{c.f}\onedot} \def\Cf{\emph{C.f}\onedot}
\def\etc{\emph{etc}\onedot} \def\vs{\emph{vs}\onedot}
\def\wrt{w.r.t\onedot} \def\dof{d.o.f\onedot}
\def\etal{\emph{et al}\onedot} \def\st{\emph{s.t}\onedot}
\pagestyle{plain}

\title{{\bf Homework Set 2, CPSC 8420, Spring 2022}} % Change to the appropriate homework number
\author{\Large\underline{Your Name}}
\date{\textbf{\Large\textcolor{red}{Due 03/17/2022, Thursday, 11:59PM EST}}} % put your name in the LastName, FirstName format
%\date{\today}

\begin{document}
	\maketitle
	

	\section*{Problem 1}
	For PCA, from the perspective of maximizing variance, please show that the solution of $\bm{\phi}$ to $\maximize \|\mX \bm{\phi}\|^2_2, \st \ \|\bm{\phi}\|_2=1$ is exactly the first column of $\mU$, where $[\mU,\mS]=svd(\mX^T\mX)$. (Note: you need prove why it is optimal than any other reasonable combinations of $\mU_i$, say $\hat{\bm{\phi}}=0.8*\mU(:,1)+0.6*\mU(:,2)$ which also  satisfies $\|\hat{\bm{\phi}}\|_2=1$.)
	


	
	\vspace{4cm}
	\section*{Problem 2}
	Why might we prefer to minimize the sum of absolute residuals instead of the residual sum of squares for some data sets? Recall clustering method $K$-means when calculating the centroid, it is to take the mean value of the data-points belonging to the same cluster, so what about $K$-medians? What is its advantage over of $K$-means? Please use a synthetic (toy) experiment to illustrate your conclusion.
		\vspace{4cm}
	
	
	
	
	\section*{Problem 3}
	Let's revisit Least Squares Problem: $\minimize \limits_{\bm{\beta}} \frac{1}{2}\|\vy-\mA\bm{\beta}\|^2_2$, where $\mA\in\R^{n\times p}$.
	\begin{enumerate}
		\item Please show that if $p>n$, then vanilla solution $(\mA^T\mA)^{-1}\mA^T\vy$ is not applicable any more.	
		\item Let's assume $\mA=[1, 2, 4;1, 3, 5; 1, 7, 7; 1, 8, 9], \vy=[1;2;3;4]$. Please show via experiment results that Gradient Descent method will obtain the optimal solution with  Linear Convergence rate if the learning rate is fixed to be $\frac{1}{\sigma_{max}(\mA^T\mA)}$, and $\bm{\beta}_0=[0;0;0]$.	
		\item Now let's consider ridge regression: $\minimize \limits_{\bm{\beta}} \frac{1}{2}\|\vy-\mA\bm{\beta}\|^2_2+\frac{\lambda}{2} \|\bm{\beta}\|^2_2$, where  $\mA,\vy,\bm{\beta}_0$ remains the same as above while learning rate is fixed to be $\frac{1}{\lambda+\sigma_{max}(\mA^T\mA)}$ where $\lambda$ varies from $0.1,1,10,100,200$, please show that Gradient Descent method with larger $\lambda$ converges faster. 
	\end{enumerate}
	\vspace{4cm}
	
	\section*{Problem 4}
	Please download the image from \url{https://en.wikipedia.org/wiki/Lenna#/media/File:Lenna_(test_image).png} with dimension $512\times512\times3$. Assume for each RGB channel data $X$, we have $[U,\Sigma,V]=svd(X)$. Please show each compression ratio and reconstruction image if we choose first $2, 5, 20, 50,80,100$ components respectively. Also please determine the best component number to obtain a good trade-off between data compression ratio and reconstruction image quality. (Open question, that is your solution will be accepted as long as it's reasonable.)
\end{document}
